%
% 北見工業大学卒業論文
%
%%%%%%%%%%%%%%%%%%%%%%%%% Preamble %%%%%%%%%%%%%%%%%%%%%%%%%%%
\documentclass[a4paper,11pt,openany]{ltjsbook}
 
\usepackage{mystyle}
\geometry{margin=19mm,top=25.4mm,bottom=25.4mm}

\renewcommand{\bibname}{参考文献}

\begin{document}

%挿入するソースコードのフォント
\lstset{
  numbers=left,
  stepnumber=1,
  numberstyle=\small\rmfamily,
  basicstyle=\small\ttfamily,
  frame={tb}
}

\renewcommand{\lstlistingname}{ソースコード}

%表紙
\thispagestyle{empty}

%学士論文
\begin{center}
 {\LARGE
 \vspace*{5cm}
 {\huge{20○○年度\ 北見工業大学\ 工学部}}\\
 {\huge{情報デザイン・コミュニケーション工学コース}}\\
 {\huge{学士論文}}\\
 \vspace{2cm}
 {\Huge{○○○に関する研究}}\\
 \vspace{6cm}
 {\huge 数理波動システム研究室}\\
 {\huge 著者}\\
 {\huge 指導教員\ \ 杉坂純一郎}\\
 }
\end{center}

%修士論文
% \begin{center}
%     {\LARGE
%     \vspace*{5cm}
%     {\huge{20○○年度\ 北見工業大学大学院\ 工学専攻}}\\
%     {\huge{情報通信工学プログラム}}\\
%     {\huge{修士論文}}\\
%     \vspace{2cm}
%     {\Huge{○○に関する研究}}\\
%     \vspace{6cm}
%     {\huge 数理波動システム研究室}\\
%     {\huge 著者}\\
%     {\huge 指導教員\ \ 杉坂純一郎}\\
%     }
% \end{center}

\newpage
\thispagestyle{empty}
\begin{center}
 \vspace{2cm}
\end{center}

\frontmatter

%目次
\tableofcontents

\mainmatter

%本文
\chapter{序論}
%数式
%\begin{align}
%    f(x) &= ax^2+bx+c_1 %&の位置で各行の数式位置をそろえる
%    \label{funcf} %式のラベル
%    \\  %改行
%    g(x) &= x+c_2
%    \label{funcg}
%\end{align}
%%式の参照:
%式(\ref{funcf})は関数$f(x)$を、式(\ref{funcg})は関数$g(x)$を表している。
%ベクトルは太字:$\vE$、変数はイタリック:$x$、それ以外はローマン体:$\mathrm{S}_1$
%\begin{align}
%    y = \tpT{\vw}\vx
%\end{align}
%単位を含む数値:$\SI{0.47}{\micro\farad}$、$\SI{3.1}{\kilo\metre}$
%%図の挿入:
%\inputfigure{abc.pdf}{ここに図の説明を書く}
%\par
%%\refは図番号に自動的に置き換わる:
%図\ref{abc.pdf}にサンプル図を示す。
%
%\begin{table}[htbp]
%    \caption{表のキャプション}
%    \label{table:tableSample}
%    \centering
%     \begin{tabular}{clll} %cは中央、lは左寄せ、rは右寄せ
%      \hline
%      AAA & BBB & CCC & DDD \\
%      \hline
%      123 & $x$ & $y$ & aaa \\
%      456 & $a$ & $b$ & bbb \\
%      \hline
%     \end{tabular}
%\end{table}
%%\refは表番号に自動的に置き換わる:
%表\ref{table:tableSample}にサンプル図を示す。
%\par
%文献\cite{sugisaka2020lensless}引用。
%\par
%ソースコード\ref{code_C}はC言語のコード。
%\begin{lstlisting}[caption={C言語},label={code_C}]
%#include <stdio.h>
%int main()
%{
%    printf("Hello World!\n");
%    return 0;  /* 正常終了 */
%}
%\end{lstlisting}
\nocite{*}
\section{研究背景}
\section{研究目的}
\section{本論文の構成}

\chapter{F{\ttfamily\#}}\label{chap:Fsharp}
\section{関数定義}
\section{クラス定義}
\section{関数定義}
\section{クラス定義}
\section{モジュール}

\chapter{関連言語}\label{chap:Related}
\input{Related_ang.tex}
\section{HTML}
\section{JavaScript}

\chapter{実装機能}\label{chap:Features}
\section{アニメーション描画領域}
\subsection{JavaScriptによる時間制御処理}
aaaaaaaaaaaaaa
\subsection{HTMLによる描画領域設定}
\subsection{ボタン表示位置の設定}
\section{ボタン効果}
\subsection{startボタン}
\subsection{resetボタン}
\section{図形}
\subsection{生成と表示}
\subsection{スタイル制御}
\subsection{時間に応じた変形・移動}
\subsection{複数図形の同時制御}
\section{画像}
\section{テキスト}
\section{アニメーション図説フレームワークの機能}
\section{アニメーション描画領域}
\subsection{JavaScriptによる時間制御処理}
\subsection{HTMLによる描画領域設定}
\subsection{ボタン表示位置の設定}
\section{ボタン効果}
\subsection{startボタン}
\subsection{resetボタン}
\section{図形}
\subsection{生成と表示}
\subsection{スタイル制御}
\subsection{時間に応じた変形・移動}
\subsection{複数図形の同時制御}
\section{画像}
\section{テキスト}
\section{アニメーション図説フレームワークの機能}

\chapter{結論}\label{chap:conclusion}
\input{Conclusion.tex}

\appendix

%付録
\chapter{コード集}
%本文に載せるまでもない公式やソースコードなどを書く。
%ソースコード内の変数を本文に書くときは\texttt{projectname}のようにフォントを変える。
%\lstinputlisting[breaklines=true]{test.fsx}

\backmatter

%謝辞
\chapter*{謝辞}
指導教員、共同研究者、協力者へのお礼、使った予算など。
敬称例)\\
教員→○○○○教授、○○○○准教授、○○○○助教\\
研究員→○○○○研究員\\
上級生→○○○○先輩\\
同級生・下級生→○○○○君\\
その他研究者(博士取得者)→○○○○博士\\
その他→○○○○氏\\
感謝致します。
最後に、大学在籍時から○年の長きにわたり研究する機会を著者に与えてくれた家族、友人に心よりの感謝の意を表します。

\rightline{20○○ 年2月 (自分の名前)}
 

%参考文献
\bibliography{reference}
\bibliographystyle{ieicetr}

\end{document}
