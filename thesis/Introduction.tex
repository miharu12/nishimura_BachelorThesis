%数式
%\begin{align}
%    f(x) &= ax^2+bx+c_1 %&の位置で各行の数式位置をそろえる
%    \label{funcf} %式のラベル
%    \\  %改行
%    g(x) &= x+c_2
%    \label{funcg}
%\end{align}
%%式の参照:
%式(\ref{funcf})は関数$f(x)$を、式(\ref{funcg})は関数$g(x)$を表している。
%ベクトルは太字:$\vE$、変数はイタリック:$x$、それ以外はローマン体:$\mathrm{S}_1$
%\begin{align}
%    y = \tpT{\vw}\vx
%\end{align}
%単位を含む数値:$\SI{0.47}{\micro\farad}$、$\SI{3.1}{\kilo\metre}$
%%図の挿入:
%\inputfigure{abc.pdf}{ここに図の説明を書く}
%\par
%%\refは図番号に自動的に置き換わる:
%図\ref{abc.pdf}にサンプル図を示す。
%
%\begin{table}[htbp]
%    \caption{表のキャプション}
%    \label{table:tableSample}
%    \centering
%     \begin{tabular}{clll} %cは中央、lは左寄せ、rは右寄せ
%      \hline
%      AAA & BBB & CCC & DDD \\
%      \hline
%      123 & $x$ & $y$ & aaa \\
%      456 & $a$ & $b$ & bbb \\
%      \hline
%     \end{tabular}
%\end{table}
%%\refは表番号に自動的に置き換わる:
%表\ref{table:tableSample}にサンプル図を示す。
%\par
%文献\cite{sugisaka2020lensless}引用。
%\par
%ソースコード\ref{code_C}はC言語のコード。
%\begin{lstlisting}[caption={C言語},label={code_C}]
%#include <stdio.h>
%int main()
%{
%    printf("Hello World!\n");
%    return 0;  /* 正常終了 */
%}
%\end{lstlisting}
\nocite{*}